\chapter{Conclusion \& outlook}
\label{chap:conclusion}
\resetallacronyms

\vspace{1em}

\section{Conclusion}
\label{sec:conclusion}

The goal of this thesis work was to demonstrate that multistage
oil-free variable-speed domestic heat pumps powered by twin-stage
radial compressors are feasible and demonstrate a potential. The
feasibility and the potential of the technology has been demonstrated
successfully.

\vspace{1em}

\subsection{Feasible}
\label{sec:concl-feasible}

The stable \OP{} detailed in \cref{sec:awp-perfs,chap:exp-details}
demonstrates that multistage oil-free variable-speed domestic heat
pumps powered by twin-stage radial compressors are
feasible. Particularly, the \OP{} A-7.0/W35.6, presented in
\cref{sec:awp-exp-details-A-7.0/W35.6}, demonstrates the feasibility
of practical domestic heat pumps for space heating, as this \OP{} is a
typical domestic heat pump \OP{} for floor heating
technologies. Higher temperature lifts need to be reached and tested,
but the concept has already demonstrated its feasibility.

\vspace{1em}

\subsection{Promising performance}
\label{sec:concl-potential}

The performance reached with the \AWP{} is considered very promising
and constitutes a breakthrough in the domain. The performance of the
\AWP{}, being a bit lower than the one of the devices on the market,
with no optimization of the circuits design, with manual control, with
a non-optimized refrigerant charge, and with numerous issues and
avoidable energy losses, are already a great achievement. Accounting
the numerous design issues which can be improved or solved in those
first prototypes, the performance is really quite promising and shows
that the radial compressors rotating on gas bearings technology is
already a success.

\vspace{1em}

\subsection{Many challenging improvements needed}
\label{sec:concl-improvements}

Of course, the challenges are still numerous before domestic heat
pumps equipped with radial compressors reach the market. The issues to
be solved are substantial but paths of solutions have already been
offered all along the chapters relative to the prototypes. So far, no
unsolvable problem could be identified. Control strategies have been
proposed and need to be implemented and tested. Defrosting and cycle
inversion are still important possible issues, but strategies and
design of key components like the economizer and the first stage
separator already answer to a good share of the potential problems
that could occur. Consequently, what is most needed is further
research, development, and tests to bring this technology to its full
and greatest potential.

\vspace{1em}

\section{Perspectives}
\label{sec:perspectives}

This thesis work opens the way to further development using the
proposals presented in the previous chapter. A really integrated
compression unit module can be developed, built, and tested, making
more efficient, more compact, more silent heat pumps built with less
raw material and using a lower refrigerant charge, a reality in the
near future. Along with oil-free heat exchangers, this compression
unit module is likely to bring further advances in the domestic heat
pump field and in the oil-free refrigeration circuits as a whole.
