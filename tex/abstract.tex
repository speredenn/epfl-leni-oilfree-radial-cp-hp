\chapter{Abstract}
\label{chap:abstract}

The world energy consumption has dramatically increased since the end
of the World War II. The main share of this energy comes from fossil
fuel combustion. The worldwide domestic sector consumes about 30\% of
the global energy supply and the biggest share of that is dedicated to
space and water heating. In this context where sustainable energy
consumption is required to limit our impact on the environment and the
climate, energy efficiency is a key factor to successfully transition
to energy sobriety. Electrically-driven heat pumps are known to be a
key technology to increase our energy efficiency. More efficient, more
compact, more silent heat pumps, built with less raw material, and
using lower refrigerant charges are needed. In electrically-driven
heat pumps, the compression process is responsible for most of the
energy losses. However, for decades now, the heat pump performance has
been stagnating, mainly because of the compression process efficiency
which has not been increasing significantly. Consequently, the
improvement of the compression process is indeed needed. A new
single-stage compression unit design has been developed, built, and
tested in a previous thesis work. This current thesis work uses a
twin-stage successor of the initial single-stage compression unit and
tests it into two oil-free domestic heat pump prototypes. This work
aims at studying the integration of the radial compression units in
domestic heat pumps and at demonstrating their feasibility and
potential. The tested prototypes are a twin-stage Air/Water domestic
heat pump and a twin-stage Brine/Water domestic heat pump. Both of
them are equipped with a twin-stage oil-free radial compression unit
rotating on gas bearings. By which, the maximum rotor speed of the
compression units is 180 krpm. Six stable operating points have been
documented with the Air/Water heat pump prototype. More notably so,
the operating point A-7/W35 has been reached and demonstrates a
coefficient of performance of 2.36 for a heating power of 10.7 kW. The
Brine/Water heat pump prototype has been used to perform partial tests
of circuit improvements, in order to solve some of the issues observed
on the Air/Water prototype. The analysis of the experimental results
uses a mass and energy balance modeling approach to improve the level
of understanding of the internal and non measurable flows in the heat
pump circuits. This model also propagates the uncertainties of the
measurements through the equations. New improvements and innovative
circuit layouts, notably at the level of the economizer, a key-component
of the two-stage heat pump circuit, are offered to integrate further
the compression unit in the heat pump circuits, and to make the whole
system more efficient, more compact, more silent, less demanding in
raw materials and reducing the refrigerant charge needed for the heat
pump cycle.

\vspace{10mm}

Keywords:~\varuskeywords
