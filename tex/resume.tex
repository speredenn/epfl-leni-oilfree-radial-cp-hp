\chapter{Résumé}
\label{chap:resume}

La consommation mondiale d'énergie a augmenté dramatiquement depuis la
fin de la seconde guerre mondiale. La part principale de cette énergie
provient de la combustion d'énergie fossile. Le secteur résidentiel
consomme environ 30\% de la production totale d'énergie et la plus
grande part de cette dernière est dédiée au chauffage des espaces de
vie et à la production d'eau chaude sanitaire. Dans ce contexte une
consommation d'énergie soutenable est requise afin de limiter notre
impact sur l'environnement et le climat, l'efficacité énergétique est
un facteur clé pour amorcer la transition vers la sobriété
énergétique. Les pompes à chaleur alimentées par électricité sont
identifiées comme une technologie clé pour augmenter notre efficacité
énergétique. Nous avons besoin de pompes à chaleur plus efficaces,
plus compactes, plus silencieuses, construites avec moins de matières
premières, et utilisant des charges en réfrigérant plus faibles. La
phase de compression est responsable de la plupart des pertes
d'énergie. Depuis des décennies maintenant, les performances des
pompes à chaleur stagnent, principalement parce que l'efficacité de la
phase de compression stagne. Par conséquent, il est nécessaire
d'améliorer l'efficacité de la phase de compression. Une nouvelle
conception d'unité compresseur mono-étagé a été développée,
construite, et testée lors d'une thèse de doctorat précédente. La
thèse présentée ici utilise une unité compresseurs bi-étagés ayant
succédé à l'unité compresseur mono-étagée initiale, et a testé l'unité
en question dans deux prototypes de pompe à chaleur domestique sans
huile. Ce travail vise à étudier l'intégration d'unités compresseurs
radiaux dans des pompes à chaleur domestiques et à démontrer la
faisabilité et le potentiel de ces dernières. Les prototypes testés
sont une pompe à chaleur domestique Air/Eau sans huile et une pompe à
chaleur domestique Saumure/Eau sans huile. Les deux prototypes sont
équipés d'une unité compresseurs bi-étagés radiaux sans huile dont
l'arbre est en rotation sur des paliers à gaz. La vitesse maximum du
rotor de ces unités est de 180000 tours/min. Six points de
fonctionnement stables ont été documentés avec le prototype
Air/Eau. Notamment, le point de fonctionnement A-7/W35 a été atteint
et démontre un coefficient de performance de 2.36 pour une puissance
chaleur de 10.7 kW. Le prototype Saumure/Eau a été utilisé pour
procéder à des tests partiels d'amélioration apportées aux circuits de
pompe à chaleur, afin d'apporter des solutions aux problèmes
rencontrés avec le prototype Air/Eau. L'analyse des données
expérimentales utilise une modélisation dont l'approche est basée sur
les équations de conservation de masse et d'énergie. Cette approche
permet d'améliorer le niveau de compréhension des débits internes ou
non-mesurables du système. Le modèle propage aussi, à travers les
équations, les incertitudes de mesure des données expérimentales. De
nouvelles améliorations et circuits de pompes à chaleur innovants,
notamment, au niveau de l'économiseur, composant-clé du circuit, sont
proposés et intègrent plus intimement l'unité compresseur dans les
circuits de la pompe à chaleur bi-étagée, et cela afin de parvenir à
des systèmes globalement plus efficaces, plus compacts, plus
silencieux, moins gourmand en terme de matières premières, et
nécessitant moins de charge de réfrigérant.

\vspace{10mm}

Mots clefs:~\varfrkeywords
